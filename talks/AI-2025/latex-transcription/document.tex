\documentclass[11pt]{article}
\usepackage{amsmath}
\usepackage{amssymb}
\usepackage{geometry}
\geometry{a4paper, margin=1in}
\usepackage{graphicx}

\title{Notes from Brutsaert: Hydrology -- An Introduction}
\author{}
\date{}

\begin{document}
	
	\maketitle
	
	\section*{Page 123}
	
	\subsection*{Energy Balance Equation}
	The energy balance equation is given as:
	\begin{equation}
		L_e E + H = Q_n \tag{4.13}
	\end{equation}
	where:
	\begin{itemize}
		\item $E$ = evaporation,
		\item $H$ = sensible heat flux into the air,
		\item $L_e$ = latent heat of evaporation (\textit{refer to Table 2.4, $2.466 \times 10^6$ J/kg}),
		\item $Q_n$ = available energy flux density.
	\end{itemize}
	
	From this equation:
	\begin{align*}
		Q_n e &= \frac{Q_n}{L_e} \quad \text{(``equivalent'' fluxes)}, \\
		H_e &= \frac{H}{L_e}.
	\end{align*}
	
	Thus, the energy equation can also be written as:
	\[
	E + H_e = Q_n e.
	\]
	
	\subsection*{Components of $Q_n$}
	The available energy flux density, $Q_n$, can be expanded as:
	\begin{equation}
		Q_n = R_n - G + L_p f + A_h + \frac{\partial W}{\partial t} \tag{4.14}
	\end{equation}
	where:
	\begin{itemize}
		\item $R_n$ = net radiation flux density,
		\item $G$ = heat flux density into the ground,
		\item $L_p f$ = thermal energy flux related to the transpiration of CO\textsubscript{2},
		\item $A_h$ = energy absorption by the canopy,
		\item $\frac{\partial W}{\partial t}$ = change in heat storage over time.
	\end{itemize}
	
	The terms of this equation represent:
	\begin{itemize}
		\item $R_n - G$: net radiation minus the heat flux into the ground,
		\item $L_p f$: contribution of CO\textsubscript{2} flux,
		\item $A_h$: atmospheric heating,
		\item $\frac{\partial W}{\partial t}$: heat storage in the system.
	\end{itemize}
	
	\subsection*{Bowen Ratio}
	The Bowen ratio, $B_o$, is defined as:
	\begin{equation}
		B_o = \frac{\gamma}{\Delta} \left[ 1 - \frac{e_a^* - \bar{e_a}}{e_s^* - \bar{e_a}} \right] \tag{4.21}
	\end{equation}
	where:
	\begin{itemize}
		\item $\gamma = \frac{C_p p}{0.622 L_e}$: the psychrometric constant,
		\item $\Delta = \frac{de_s^*}{dT}$: slope of the saturation water vapor pressure curve at $T_a$,
		\item $e_s^*$ = saturation water vapor pressure,
		\item $\bar{e_a}$ = mean vapor pressure,
		\item $e_a^*$ = actual water vapor pressure at air temperature.
	\end{itemize}
	
	The Bowen ratio describes the ratio of sensible heat flux to latent heat flux and plays a critical role in energy balance studies.
	
	\subsection*{Key Definitions and Constants}
	\begin{itemize}
		\item $e_s^*$: saturation water vapor pressure at the surface,
		\item $\bar{e_a}$: mean vapor pressure in the atmosphere,
		\item $e_a^*$: saturation water vapor pressure at air temperature,
		\item $\gamma = \frac{C_p p}{0.622 L_e}$: psychrometric constant (specific heat capacity of air, $C_p$, at constant pressure $p$),
		\item $\Delta = \frac{e_s^* - e_a^*}{T_s - T_a}$: slope of the saturation water vapor pressure curve.
	\end{itemize}
	
	\subsection*{Saturation Water Vapor Pressure Curve}
	The curve of saturation water vapor pressure as a function of temperature is critical for hydrological studies. Refer to Page 28 (Equation 2.14) and Figure 2.1 for details:
	\[
	e^*(T) \quad \text{versus temperature}.
	\]
	
	The slope of this curve, $\Delta$, is defined as:
	\[
	\Delta = \frac{de_s^*}{dT}.
	\]
	
	\begin{figure}[h!]
		\centering
		\includegraphics[width=0.6\textwidth]{example-graph-placeholder.png} % Replace with actual graph if available
		\caption{Saturation water vapor pressure curve (Figure 2.1).}
	\end{figure}
	
\end{document}
